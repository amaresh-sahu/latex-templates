


%
% *** Introduction
%

\section{Introduction}

If you're interested, take a look at one of my first papers \cite{sahu-mandadapu-pre-2017}.
\lipsum[2]



%
% *** First Section
%

\section{First Section} \label{sec:first_section}

\lipsum[3]



%
% *** First Subsection
%

\subsection{First Subsection}

\lipsum[4]



%
% *** First Sub-subsection
%

\subsubsection{First Sub-subsection}

\lipsum[5]



%
% *** Paragraph
%

\paragraph{Paragraph}

\lipsum[6]



%
% *** Compiling the PDF
%

\section{Compiling the PDF}

I use the command line to compile my PDF files.
Run
\begin{verbatim}
	xelatex paper.tex
\end{verbatim}
to compile the PDF.
Every time you modify the \verb+refs.bib+ bibliography file, you will need to run
\begin{verbatim}
	xelatex paper.tex
	bibtex paper.aux
	xelatex paper.tex
	xelatex paper.tex
\end{verbatim}
to correctly format the bibliography.
We include bibtex formatting, rather than the more powerful and user-friendly biber back-end, because Arxiv submissions require a bibtex back-end.



%
% *** Code Formatting
%

\section{Code Formatting}

Display algorithms as follows.

\begin{lstlisting}[caption={\texttt{C++} pseudocode of an algorithm}]
 // mesh and basis function calculations
 generate_mesh(); generate_basis_functions();

 for (time_index = 0; time_index < num_time_steps; ++time_index) {

	initialize_u_vector(); initialize_delta_u();

	while (norm(delta_u) > newton_tolerance) {
		// initialize global residual vector and stiffness matrix
		initialize_r_vector(); initialize_K_matrix();

		assemble_K_matrix();
		assemble_r_vector();

		// apply boundary conditions
		apply_boundary_conditions(K_matrix, r_vector);

		solve_delta_u(delta_u, K_matrix, r_vector);
		u_vector += delta_u;
	}

	output_u_vector();
 }
\end{lstlisting}

