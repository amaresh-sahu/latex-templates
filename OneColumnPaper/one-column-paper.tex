\documentclass[11pt,notitlepage]{article}
\usepackage{amsmath}
\usepackage{amssymb}
\usepackage{amsthm}
\usepackage{array}
\usepackage{bm}
\usepackage[font={footnotesize}]{caption}
\usepackage{enumerate}
\usepackage{fancyhdr}
\usepackage{fixltx2e}
\usepackage{float}
\usepackage[T1]{fontenc}
\usepackage{graphicx}
\usepackage[colorlinks=true,citecolor=MidnightBlue,linkcolor=BrickRed,breaklinks=true]{hyperref}
\usepackage{mathrsfs}
\usepackage{titlesec}
\usepackage{wrapfig}
\usepackage[usenames,dvipsnames]{xcolor}
\usepackage{xy}

\pdfpagewidth 8.5in
\pdfpageheight 11in 
\topmargin 0in
\evensidemargin 0in
\oddsidemargin 0in
\headheight 15pt
\headsep 0.25in
\topskip 0in
\textheight 8.5in
\textwidth 6.5in
\footskip 0.5in
\parskip 0in
\parindent 0.3in


\pagestyle{fancy}
\lhead{\today}
\chead{One-Column Template}
\rhead{Amaresh Sahu}


\title{\bfseries{\Large{\sffamily A One-Column Latex Template}}}
\author{\large{\sffamily Amaresh Sahu}}
\date{\normalsize \sffamily \today}



% Section Header Formatting
\titleformat{\section}
{\normalfont \large \bfseries}
{\thesection.\,}
{0pt}
{}

\titleformat{\subsection}
{\normalfont \bfseries \itshape}
{\thesubsection\,\,}
{0pt}
{}

\renewcommand{\thesubsubsection}{\arabic{subsubsection}}

\titleformat{\subsubsection}
{\normalfont \itshape}
{\S \thesubsubsection.\,\,}
{0pt}
{}





\begin{document}
\thispagestyle{empty}



%
% *** Title
%

\maketitle



%
% *** Abstract
%

\renewcommand{\abstractname}{}
\begin{abstract} \sffamily
	Begin abstract here.
	This will naturally be less wide compared to the rest of the page.
	You can adjust the space after the abstract, as well as whether or not you print the word ``Abstract'' in the \verb+.tex+ file.
	\vspace{20pt}
\end{abstract}



%
% *** Introduction
%

\section{Introduction}

Simply start typing here.
Starting on a new line in the \verb+.tex+ file doesn't start a new line in the pdf document - rather, you'll need to specifically tell latex you want a new line.
\bigskip

This can be done like this.
Notice Latex automatically takes care of the paragraph tabbing.
You can also add in equations by saying
\begin{equation} \label{eq:intro-equation}
	e^{i\pi} + 1 = 0,
\end{equation}
and reference them with \eqref{eq:intro-equation}.



%
% *** Mathematical Preliminaries
%

\section{Mathematical Preliminaries} \label{sec:math-prelims}

There are several options into how to structure your paper.
For example, sections.
The current section we are in, Section \ref{sec:math-prelims}, is about mathematical preliminaries



%
% *** Geometry
%

\subsection{Geometry}

We can use subsections.



%
% *** Differential Geometry
%

\subsubsection{Differential Geometry}

Finally, we can use ``subsubsections'' for even more granular levels of detail.



%
% *** Compiling the PDF
%

\section{Compiling the PDF}

I use the command line to compile my PDF files.
Run
\begin{verbatim}
	xelatex one-column-paper.tex
\end{verbatim}
to compile the PDF.
Every time you modify the \verb+refs.bib+ bibliography file, you will also be prompted to run
\begin{verbatim}
	bibtex one-column-paper.aux
\end{verbatim}
to correctly format the bibliography.
We include bibtex formatting, rather than the more powerful and user-friendly biber back-end, because Arxiv submissions require a bibtex back-end.



%
% *** Misc
%

\section{Misc}

Note that starting from the second page of the paper onwards, pages have a heading at the top of the page.
We can cite a paper as long as it in our \verb+refs.bib+ file.
An interesting paper on microfluidic fuel cells could be found in \cite{vigolo-stone-rsc-2014}.




%
% *** Bibliography
%

\newpage
\addcontentsline{toc}{section}{References}
\bibliography{../refs}
\bibliographystyle{abbrv}




\end{document}
