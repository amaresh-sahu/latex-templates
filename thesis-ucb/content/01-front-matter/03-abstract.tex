


\vspace*{0.5cm}
\begin{center}
	\textbf{\large Abstract}\\[34pt]
	Irreversible Thermodynamics and Hydrodynamics\\[3pt]
	of Biological Membranes\\[12pt]
	by\\[12pt]
	Amaresh Sahu\\[14pt]
	Doctor of Philosophy in Chemical Engineering\\[4pt]
	University of California, Berkeley\\[4pt]
	Professor Kranthi K.\ Mandadapu, Chair
\end{center}
\vspace{20pt}

This thesis is concerned with developing a wholistic description of biological membranes: fascinating materials that make up the boundary of the cell, as well as many of the cell's internal organelles.
Our formulation of the theory of such materials relies on two well-known concepts: differential geometry and irreversible thermodynamics.
The setting of differential geometry allows us to describe curves and surfaces, which in this case are embedded in the three-dimensional Euclidean space, while irreversible thermodynamics provides a theoretical framework to develop constitutive relations between the various thermodynamic forces and fluxes in a system.
Both concepts are well-known, and are reviewed in Part \ldots



