


%
% *** Important Chapter
%

\chapter{Important Chapter} \label{chap_important}
\setcounter{footnote}{0}

\begin{displayquote}
	\fontfamily{bch}\selectfont
	\textit{%
		Classical thermodynamics has solved the problem of the competition
		between randomness and organization for equilibrium situations.
		How then is it possible to extend these results to dissipative systems?
		What part of the energy flow may be used to create and maintain some
		structure in such systems?
	}\\
	\rule{0mm}{1pt}\hfill---Ilya R.\ Prigogine~\footcite{prigogine}
\end{displayquote}

I often like to begin thesis chapters with a quote.
Note that if you use the \verb+\footcite{}+ command, then the citation will appear as a footnote, and will also be reported \textit{at the end of each chapter.}
For this reason, the file \verb+bib.tex+ should be included after every chapter in the file \verb+thesis.tex+ with the command \verb+

\phantomsection
\sectionmark{\MakeLowercase{References}}
\renewcommand{\refname}{\textsc{\MakeLowercase{References}}}
\section*{References}
\addcontentsline{toc}{section}{References}
\printbibliography[heading=none]

+.
You can also use the \verb+\footnote{}+ command, and then the \verb+\cite{}+ command within the footnote, as in the following example which produces the footnote below:%
\footnote{Consider the scaling analysis by \cite{sahu-mandadapu-pre-2020}}
\\[8pt]
\verb+\footnote{Consider the scaling analysis by \cite{sahu-mandadapu-pre-2020}}+
\\[8pt]
Note that you can include arXiv links in your citations!
The bibliography is compiled by running the command \verb+biber thesis.bcf+ in the command line.



%
% *** The equation environment
%

\section{The equation environment} \label{sec_equation_environment}

Many custom commands are provided in the \verb+custom-commands.tex+ file.
The following equation was obtained by writing
\verb+\bmnabla \bmcdot \bmv \, = \, 0 ~.+
\begin{equation}
	\bmnabla \bmcdot \bmv
	\, = \, 0
	~.
\end{equation}
In general, I recommend adding additional spaces in math mode, to create a more aesthetic result.
The command \verb+\+ creates a large space, the command \verb+\,+ creates a medium space, and the custom command \verb+\mke+ (equivalently \verb+\mkern1mu+) makes a small space.
Additionally, the command \verb+~+ creates a very large space.



%
% *** The additional environments
%

\section{The additional environments} \label{sec_additional_environments}

Here, we show two additional features.  
You can construct an `Example' environment (currently, this can handle page breaks, but does not maintain nice spacing):

\begin{example}[An Example Environment]
	Consider an example, which you would like to present to the reader.
	I like the following equation:
	\begin{equation*}
		\mathrm{e}^{i \pi}
		\, + \, 1
		\, = \, 0
		~.
	\end{equation*}
\end{example}


\noindent Also, you can display algorithms as follows:

\begin{lstlisting}[caption={\texttt{C++} pseudocode of an algorithm}]
 // mesh and basis function calculations
 generate_mesh(); generate_basis_functions();

 for (time_index = 0; time_index < num_time_steps; ++time_index) {

	initialize_u_vector(); initialize_delta_u();

	while (norm(delta_u) > newton_tolerance) {
		// initialize global residual vector and stiffness matrix
		initialize_r_vector(); initialize_K_matrix();

		assemble_K_matrix();
		assemble_r_vector();

		// apply boundary conditions
		apply_boundary_conditions(K_matrix, r_vector);

		solve_delta_u(delta_u, K_matrix, r_vector);
		u_vector += delta_u;
	}

	output_u_vector();
 }
\end{lstlisting}

\noindent Both the example and algorithm environments can be modified in the \texttt{thesis.sty} file.
\newpage



