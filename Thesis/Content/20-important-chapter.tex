


%
% *** Important Chapter
%

\chapter{Important Chapter} \label{chap:chap_important}

If you desire, you can split your thesis into ``parts'' with the \verb+\part{}+ command.
Some other cool things you can do:

\begin{example}[An Example Environment]
	Consider an example, which you would like to present to the reader.
	I like the following equation:
	\begin{equation*}
		\mathrm{e}^{i \pi} + 1 = 0
		~.
	\end{equation*}
\end{example}

Also, you can display algorithms as follows:

\begin{lstlisting}[caption={\texttt{C++} pseudocode of an algorithm}]
 // mesh and basis function calculations
 generate_mesh(); generate_basis_functions();

 for (time_index = 0; time_index < num_time_steps; ++time_index) {

	initialize_u_vector(); initialize_delta_u();

	while (norm(delta_u) > newton_tolerance) {
		// initialize global residual vector and stiffness matrix
		initialize_r_vector(); initialize_K_matrix();

		assemble_K_matrix();
		assemble_r_vector();

		// apply boundary conditions
		apply_boundary_conditions(K_matrix, r_vector);

		solve_delta_u(delta_u, K_matrix, r_vector);
		u_vector += delta_u;
	}

	output_u_vector();
 }
\end{lstlisting}

Both the example and algorithm environments can be modified in the \texttt{thesis.sty} file.

