


%
% *** Introduction
%

\section{Introduction}

Simply start typing here.
Starting on a new line in the \verb+.tex+ file doesn't start a new line in the pdf document - rather, you'll need to specifically tell latex you want a new line.
\bigskip

This can be done like this.
Notice Latex automatically takes care of the paragraph tabbing.
You can also add in equations by saying
\begin{equation} \label{eq:intro-equation}
	e^{i\pi} + 1 = 0,
\end{equation}
and reference them with \eqref{eq:intro-equation}.



%
% *** Mathematical Preliminaries
%

\section{Mathematical Preliminaries} \label{sec:math-prelims}

There are several options into how to structure your paper.
For example, sections.
The current section we are in, Section \ref{sec:math-prelims}, is about mathematical preliminaries



%
% *** Geometry
%

\subsection{Geometry}

We can use subsections.



%
% *** Differential Geometry
%

\subsubsection{Differential Geometry}

Finally, we can use ``subsubsections'' for even more granular levels of detail.



%
% *** Compiling the PDF
%

\section{Compiling the PDF}

I use the command line to compile my PDF files.
Run
\begin{verbatim}
	xelatex one-column-paper.tex
\end{verbatim}
to compile the PDF.
Every time you modify the \verb+refs.bib+ bibliography file, you will also be prompted to run
\begin{verbatim}
	bibtex one-column-paper.aux
\end{verbatim}
to correctly format the bibliography.
We include bibtex formatting, rather than the more powerful and user-friendly biber back-end, because Arxiv submissions require a bibtex back-end.



%
% *** Misc
%

\section{Misc}

Note that starting from the second page of the paper onwards, pages have a heading at the top of the page.
We can cite a paper as long as it in our \verb+refs.bib+ file.
An interesting paper on microfluidic fuel cells could be found in \cite{vigolo-stone-rsc-2014}.

